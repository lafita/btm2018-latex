\documentclass[inputenc]{beamer}

\usepackage{graphicx}
\usepackage{url}

% Color blue for the URL links
\hypersetup{
    colorlinks=true,
    linkcolor=blue,
    urlcolor=purple,
}

% This adds the table of contents before each section
\AtBeginSection[]
{
  \begin{frame}<beamer>
  \frametitle{Outline}
  \tableofcontents[currentsection]
  \end{frame}
}
  
\title[Introduction to LaTeX]{Introduction to \LaTeX{}}
\author{Aleix Lafita}
\institute{Alex Bateman group (EMBL-EBI)}
\date{16th October 2018}

\begin{document}

\begin{frame}
  \titlepage
\end{frame}

\begin{frame}{Outline}
    \tableofcontents
\end{frame}

%%%%%%%%%%%%%%%%%%%%%%%%%%%%%%%%%%%%%%%%%%%%%%%%%%%%%%%%%%
\section{Introduction}

\begin{frame}{Introduction}

TeX is a typesetting language. Instead of visually formatting your text, you enter your manuscript text intertwined with TeX commands in a plain text file. You then run TeX to produce formatted output, such as a PDF file. Thus, in contrast to standard word processors, your document is a separate file that does not pretend to be a representation of the final typeset output, and so can be easily edited and manipulated.

\end{frame}

\begin{frame}{How do Pros pronounce it?}
    
    
    
    % That said, I will probably pronounce it wrongly during the presentation...
    
\end{frame}

\begin{frame}{The Levels of TeX and terms  \footnote{\url{https://www.tug.org/levels.html}}}
    
    \begin{itemize}
        \item \textbf{Format:} the syntax you type in (TeX, LaTeX)
        \item \textbf{Engines:} what creates the PDF from your text file (pdfLaTeX, XeLaTeX, LuaTeX)
        \item \textbf{Distributions:} TeX Live, MiKTex
    \end{itemize}
    
   
    
\end{frame}

\begin{frame}{Motivation}
    
    Advantages / disadvantages
    
\end{frame}

\begin{frame}{Cool demos}
    
    formula
    image
    table
    
\end{frame}


%%%%%%%%%%%%%%%%%%%%%%%%%%%%%%%%%%%%%%%%%%%%%%%%%%%%%%%%%%
\section{Hand on}

\subsection{Your first LaTeX document}

\begin{frame}{Your first LaTeX document}
    
    We will use Overleaf to create your first document.
    
\end{frame}

\subsection{The ultimate challenge}

\begin{frame}{The ultimate challenge}
    
    Collaborate on the same thesis document, each of you should write a chapter.
    
\end{frame}

%%%%%%%%%%%%%%%%%%%%%%%%%%%%%%%%%%%%%%%%%%%%%%%%%%%%%%%%%%
\section{Materials}

\begin{frame}{Materials}

\begin{itemize}
    \item Most of the introduction is based on \textit{The Not So Short Introduction to LATEX}:
    
    \url{http://tug.ctan.org/info/lshort/english/lshort.pdf}. The document covers many other advanced topics.
    \item \textit{Thesis writting in LaTeX} created by predocs at the EBI: \url{https://github.com/EBI-predocs/latex-thesis}
    \item \textit{TeX Live}: to download LaTeX to your computer \url{https://www.tug.org/texlive}
\end{itemize}

Trivial reminder: if you need help, the internet is your friend!

\end{frame}

\begin{frame}{Get to the source}
    
    This presentation was made in LaTeX.
    In the unlikely case you want to see the code or use it as an example, here it is:
    
    \begin{itemize}
        \item Overleaf: \url{https://v2.overleaf.com/read/xbmthmbvvbhb}
        \item GitHub: \url{https://github.com/lafita/btm2018-latex}
    \end{itemize}
    
\end{frame}


\end{document}
